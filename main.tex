\documentclass{article}
\usepackage[utf8]{inputenc}

\title{Gymkhana News Web app}
\author{Team-Netstats }

\usepackage{natbib}
\usepackage{graphicx}
\usepackage{hyperref}

\begin{document}

\maketitle

\section{Aim of the project}
Aim of the project is to create a web application for the Students Gymkhana. The web app should be able to run on mobile devices like an app would. The interface will be similar to that of \href{http://www.quora.com}{Quora} /
\href{http://www.twitter.com}Twitter. \\
Main function of the app will be to display the latest news and upcoming and ongoing events & highlights with real-time updates. The homepage will contain recent/top highlights. This page will also contain tabs for the councils/hobby groups/senate - all of which will have their own news page, each showing news related to the clubs of the council.\\We'll implement it via push notifications and the users will have an option to like or share it on facebook/google+.Also the notification will have the relevant links to websites and other documents. We'll also add the contact details of all the cordinators and secretaries along with their phone no,room,email id. Clicking on the email-id will open a form to directly send an email/query to that person. All the users will be authenticated via the cc userids.Right now we will focus mainly on SNT council web page. \\

\section{Front-end}
We are planning to do the front-end development using bootstrap.

\section{Back-end}
We are planning to do the back-end in python. We will be using the Comet web application model (\url{https://www.wikiwand.com/en/Comet_(programming)}) . We need to study the details of how push notifications work. For database MySQL will be required.

\section{Tentative Timeline}
\subsection{June 2, 2016}
MySQL, complete reading material related to back-end,complete the basic layout of the webapp on paper .
\subsection{June 10, 2016}
Implementation of back-end in python and start with the front end using bootstrap.
\subsection{June 15, 2016}
Complete the initial working website by completing the front end and integrating it with the back end.
\subsection{June 20,2016}
Bug fixes, implementation of extra features like open blog, making an android app or extending it for complete student gymkhana.
\subsection{June 25,2016}
Final Presentation.

\end{document}
